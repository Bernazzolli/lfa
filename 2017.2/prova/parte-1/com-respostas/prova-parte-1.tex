\documentclass[12pt,a4paper,oneside]{article}

\usepackage[utf8]{inputenc}
\usepackage[portuguese]{babel}
\usepackage[T1]{fontenc}
\usepackage{amsmath}
\usepackage{amsfonts}
\usepackage{amssymb}
\usepackage{graphicx}
\usepackage{xcolor}
\usepackage{multicol}
\usepackage{tikz}
\usetikzlibrary{automata,positioning}

% Definindo novas cores
\definecolor{verde}{rgb}{0.25,0.5,0.35}

\author{\\Universidade Federal de Goiás (UFG) - Regional  Jataí\\Bacharelado em Ciência da Computação \\Linguagens Formais e Autômatos \\Esdras Lins Bispo Jr.}

\date{20 de fevereiro de 2018}

\title{\sc \huge Prova (Parte 1)}

\begin{document}

\maketitle

{\bf ORIENTAÇÕES PARA A RESOLUÇÃO}

\small
 
\begin{itemize}
	\item A avaliação é individual, sem consulta;
	\item A pontuação máxima desta avaliação é 10,0 (dez) pontos, sendo uma das 06 (seis) componentes que formarão a média final da disciplina: quatro testes, uma prova e exercícios-bônus;
	\item A média final ($MF$) será calculada assim como se segue
	\begin{eqnarray}
		MF & = & MIN(10, S) \nonumber \\
		S & = & (\sum_{i=1}^{4} 0,2.T_i ) + 0,2.P  + EB\nonumber
	\end{eqnarray}
	em que 
	\begin{itemize}
		\item $S$ é o somatório da pontuação de todas as avaliações,
		\item $T_i$ é a pontuação obtida no teste $i$,
		\item $P$ é a pontuação obtida na prova, e
		\item $EB$ é a pontuação total dos exercícios-bônus.
	\end{itemize}
	\item O conteúdo exigido desta avaliação compreende o seguinte ponto apresentado no Plano de Ensino da disciplina: (1) Revisão de Fundamentos, (2) Autômatos Finitos Determinísticos, e (3) Autômatos Finitos Não-Determinísticos.
\end{itemize}

\begin{center}
	\fbox{\large Nome: \hspace{10cm}}
\end{center}

\newpage

\begin{enumerate}
	
	\section*{Primeiro Teste}
	
	\item (5,0 pt) {\bf [Sipser 0.5]} Se $C$ é um conjunto com $n$ elementos, quantos elementos estão no conjunto das partes de $C$? Explique sua resposta.
	
	\vspace*{0.1cm}
	
	{\color{blue} R - O conjunto das partes de $C$ têm $2^n$ elementos. O conjunto das partes de $C$ tem todos os subconjuntos de $C$. Logo, todos os subconjuntos de 0 elemento, de 1 elemento, até todos os subconjuntos de $n$ elementos são membros de $\mathcal{P}(C)$. Assim, $|\mathcal{P}(C)| = C^n_0 + C^n_1 \ldots C^n_n = 2^n$.
	}
	
	\item (5,0 pt) {\bf [Sipser 1.3]} A descrição formal de um AFD $M$ é \\
	$(\{q_1, q_2, q_3, q_4, q_5\}, \{u,d\}, \delta, q_3, \{q_3\})$ em que $\delta$ é dada pela tabela a seguir.
	
	\begin{center}
		\begin{tabular}{c|cc}
					&	$u$		&	$d$	\\
			\hline
			$q_1$	&	$q_1$ 	&	$q_2$	\\
			$q_2$	&	$q_1$ 	&	$q_3$	\\
			$q_3$	&	$q_2$ 	&	$q_4$	\\
			$q_4$	&	$q_3$ 	&	$q_5$	\\
			$q_5$	&	$q_4$ 	&	$q_5$	\\
			\hline
		\end{tabular}
	\end{center}
	
	Dê o diagrama de estados desta máquina.
	
	\vspace*{0.3cm}
	
	{\color{blue}
		
		\begin{tikzpicture}
		[shorten >=1pt,node distance=2cm,on grid,auto] 
		\node[state, initial, accepting] (q_3) {$q_3$}; 
		\node[state](q_4) [right=of q_3] {$q_4$};
		\node[state](q_5) [right=of q_4] {$q_5$};				
		\node[state] (q_2) [below=of q_3] {$q_2$};
		\node[state] (q_1) [below=of q_2] {$q_1$}; 
		\path[->] 
		(q_1)	edge [loop below] node {u} ()
			 	edge  [bend left] node {d} (q_2)
		(q_2) 	edge  [bend left] node {u} (q_1)
				edge  [bend left] node {d} (q_3)
		(q_3) 	edge  [bend left] node {u} (q_2)
				edge  [bend left] node {d} (q_4)
		(q_4) 	edge  [bend left] node {u} (q_3)
				edge  [bend left] node {d} (q_5)
		(q_5) 	edge  [bend left] node {u} (q_4)
				edge [loop right] node {d} ();
		\end{tikzpicture}
	}

	\newpage

	\section*{Segundo Teste}
	
	\item (5,0 pt) {\bf [Sipser 1.11]}  Prove que todo AFN pode ser convertido em um AFN equivalente que tenha apenas um único estado final.
	
	\vspace*{0.1cm}
	
	{\color{blue} R - Seja $N = (Q_N, \Sigma_N, \delta_N, q_N, F_N)$ um AFN qualquer. Pode-se construir o AFN $M = (Q, \Sigma, \delta, q_0, \{ q_f \})$, a partir de  $N$, de forma que $L(N) = L(M)$. Os elementos de $N$ são descritos a seguir:
		\begin{itemize}
			\item $Q = Q_N \cup \{q_f\}$;
			\item $\Sigma = \Sigma_N$;
			\item $\delta(q,a) = \left\{\begin{array}{cl}
			\delta_N(q,a) \cup \{ q_f \}, 			& \text{se } q \in F_N \text{ e } a=\epsilon\\
			\delta_N(q,a), 		& \text{caso contrário.}\\
			\end{array} \right.$\\
			em que $q \in Q$ e $a \in \Sigma$;
			\item $q_0 = q_N$;
			\item $q_f$ é o único estado final $\blacksquare$
			
		\end{itemize}
	}
	
	\item (5,0 pt) {\bf [Sipser 1.14 (b)]} Mostre através de um exemplo que, se $M$ é um AFN que reconhece a linguagem $C$, trocar os seus estados simples pelos finais (e vice-versa) não garante necessariamente que o novo AFN reconhece o complemento de $C$. A classe de linguagens reconhecidas por AFNs é fechada sob a operação de complemento? Justifique as suas respostas.
	
	\vspace*{0.3cm}
	
	{\color{blue} {\bf Resposta:} Seja $M$ o AFN a seguir
		
		\begin{tikzpicture}
		[shorten >=1pt,node distance=2cm,on grid,auto] 
		\node[state,initial] (q_1)   {$q_1$}; 
		\node[state, accepting] (q_2) [right=of q_1] {$q_2$}; 
		\path[->] 
		(q_1)	edge [loop above] node {0,1} ()
		edge  [bend left] node {1} (q_2);
		\end{tikzpicture}
		 \ \ Logo $L(M) = \Sigma^*1$. Entretanto, o novo AFN $N$, construído conforme sugerido, seria
		
		\begin{tikzpicture}
		[shorten >=1pt,node distance=2cm,on grid,auto] 
		\node[state,initial, accepting] (q_1)   {$q_1$}; 
		\node[state] (q_2) [right=of q_1] {$q_2$}; 
		\path[->] 
		(q_1)	edge [loop above] node {0,1} ()
		edge  [bend left] node {1} (q_2);
		\end{tikzpicture}
		de forma que $L(N) \not= \overline{L(M)}$, pois $L(N) = \Sigma^*$.
		
		Apesar disto, a classe de linguagens reconhecidas por AFNs é fechada sob a operação de complemento. Isto é verdade, pois todo AFN pode ser convertido para um AFD (e vice-versa). E uma linguagem reconhecida por um AFD é uma linguagem regular. 
		
		Desta forma, a classe de linguagens reconhecidas por AFNs é a classe de linguagens regulares. E a classe de linguagens regulares é fechada sob a operação de complemento. E, por este motivo, que a classe de linguagens reconhecidas por AFNs é fechada sob a operação de complemento.
	}

\end{enumerate}

\end{document}