\documentclass[12pt,a4paper,oneside]{article}

\usepackage[utf8]{inputenc}
\usepackage[portuguese]{babel}
\usepackage[T1]{fontenc}
\usepackage{amsmath}
\usepackage{amsfonts}
\usepackage{amssymb}

\usepackage{multirow}
\usepackage{array,graphicx}

\usepackage{xcolor}
% Definindo novas cores
\definecolor{verde}{rgb}{0.25,0.5,0.35}
\definecolor{jpurple}{rgb}{0.5,0,0.35}

\author{\\Universidade Federal de Goiás (UFG) - Regional Jataí\\Bacharelado em Ciência da Computação \\Linguagens Formais e Autômatos - 2017.2 \\Prof. Esdras Lins Bispo Jr.}
\date{}

\title{
	\sc \huge Listas de Exercícios
	\\{\tt Versão 3.0}
}

\begin{document}

\maketitle

\section{Livro de Referência}
	\begin{itemize}
		\item SIPSER, M. {\bf Introdução à Teoria da Computação}, 2a Edição, Editora Thomson Learning, 2011. \color{blue}{\bf Código Bib.: [004 SIP/int]}.
	\end{itemize}
	
\section{Listas de Exercícios}

\begin{enumerate}

	\subsection{Teste 1}
	\item[] {\bf Lista de Exercícios 01:} 0.1 ao 0.12.
	\item[] {\bf Lista de Exercícios 02:} 1.1 ao 1.6;
	\subsection{Teste 2}
	\item[] {\bf Lista de Exercícios 03:} 1.7 ao 1.11, 1.15, 1.16 e 1.31; 
	\subsection{Teste 3}
	\item[] {\bf Lista de Exercícios 04:} 1.17 ao 1.22, 1.28;
	%Lista de Exercícios 04: 2.1, 2.3, 2.4, 2.6 e 2.8.
	
\end{enumerate}

\end{document}