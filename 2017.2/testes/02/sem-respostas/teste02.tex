\documentclass[12pt,a4paper,oneside]{article}

\usepackage[utf8]{inputenc}
\usepackage[portuguese]{babel}
\usepackage[T1]{fontenc}
\usepackage{amsmath}
\usepackage{amsfonts}
\usepackage{amssymb}
\usepackage{graphicx}
\usepackage{xcolor}
\usepackage{multicol}
% Definindo novas cores
\definecolor{verde}{rgb}{0.25,0.5,0.35}

\author{\\Universidade Federal de Goiás (UFG) - Regional  Jataí\\Bacharelado em Ciência da Computação \\Linguagens Formais e Autômatos \\Esdras Lins Bispo Jr.}

\date{27 de novembro de 2017}

\title{\sc \huge Segundo Teste}

\begin{document}

\maketitle

{\bf ORIENTAÇÕES PARA A RESOLUÇÃO}

\small
 
\begin{itemize}
	\item A avaliação é individual, sem consulta;
	\item A pontuação máxima desta avaliação é 10,0 (dez) pontos, sendo uma das 06 (seis) componentes que formarão a média final da disciplina: quatro testes, uma prova e exercícios-bônus;
	\item A média final ($MF$) será calculada assim como se segue
	\begin{eqnarray}
		MF & = & MIN(10, S) \nonumber \\
		S & = & (\sum_{i=1}^{4} 0,2.T_i ) + 0,2.P  + EB\nonumber
	\end{eqnarray}
	em que 
	\begin{itemize}
		\item $S$ é o somatório da pontuação de todas as avaliações,
		\item $T_i$ é a pontuação obtida no teste $i$,
		\item $P$ é a pontuação obtida na prova, e
		\item $EB$ é a pontuação total dos exercícios-bônus.
	\end{itemize}
	\item O conteúdo exigido desta avaliação compreende o seguinte ponto apresentado no Plano de Ensino da disciplina: (2) Autômatos Finitos Determinísticos, e (3) Autômatos Finitos Não-Determinísticos.
\end{itemize}

\begin{center}
	\fbox{\large Nome: \hspace{10cm}}
\end{center}

\newpage

\begin{enumerate}
	
	\section*{Segundo Teste}
	
	\item (5,0 pt) Dê o diagrama de estados dos {\bf AFNs} que reconhecem as seguintes linguagens. Admita em todos os itens que o alfabeto é  $\{0,1\}$.
		\begin{enumerate}
			\item {\bf [Sipser 1.7 (c)]} (1,5 pt) \\$\{\omega$ | $\omega$ contém um número par de {\sf 0}s, ou contém exatamente dois {\sf 1}s$\}$.
			\item {\bf [Sipser 1.9 (a)]} (2,0 pt) $A \circ B$, em que \\$A = \{\omega$ | o comprimento de $\omega$ é no máximo 5$\}$ e
			\\$B = \{\omega$ | toda posição ímpar de $\omega$ é um {\sf 1}$\}$.
			\item {\bf [Sipser 1.10 (b)]} (1,5 pt) $A^*$, em que \\$A = \{\omega$ | $\omega$ contém ao menos dois {\sf 0}s e no máximo um {\sf 1}s$\}$.
		\end{enumerate}
	
	\item (5,0 pt) {\bf [Sipser 1.31]} Para qualquer cadeia $\omega = \omega_1 \omega_2 \ldots \omega_n$, o reverso de $\omega$, chamado de $\omega^{\mathcal{R}}$, é a cadeia $\omega$ em ordem reversa, $\omega_n \ldots \omega_2 \omega_1$. Para qualquer linguagem $A$, faça que $A^{\mathcal{R}} = \{ \omega^{\mathcal{R}}$ | $\omega \in A\}$. Mostre que se $A$ é regular, então $A^{\mathcal{R}}$ também é regular.

\end{enumerate}

\end{document}